% senior_thesis-proposal.tex
% CMPSC 580, Spring 2019
%
%
% This document provides a sample senior thesis proposal template for use
% by students in Allegheny's CS and Applied Computing programs.
%
%   *******************************************************************
%   * LOOK FOR BLOCK COMMENTS SUCH AS THIS ONE FOR AN EXPLANATION OF  *
%   * THIS DOCUMENT AND HOW TO MODIFY IT FOR YOUR OWN PROPOSAL!       *
%   *                                                                 *
%   * ANY LINE BEGINNING WITH A "%" IS A LATEX COMMENT AND IS IGNORED *
%   * BY THE LATEX PROCESSOR. YOU ARE ENCOURAGED TO COMMENT YOUR OWN  *
%   * LATEX CODE.                                                     *
%   *******************************************************************

%   ********************************************************************
%   * THE FIRST SECTION OF THE LATEX FILE IS THE "PREAMBLE." IT        *
%   * INSTRUCTS LATEX TO IMPORT SPECIAL PACKAGES FOR THINGS LIKE       *
%   * INCLUDING FIGURES, DOUBLE-SPACING, COLORED TEXT, ETC.            *
%   * DEPENDING ON YOUR NEEDS, YOU MAY FIND IT NECESSARY TO USE PACK-  *
%   * AGES THAT ARE NOT INCLUDED IN THIS TEMPLATE. SIMPLY IMITATE THE  *
%   * "\usepackage{...}" COMMANDS SHOWN BELOW.                         *
%   ********************************************************************

%   ********************************************************************
%   * BEGINNING OF PREAMBLE:                                           *
%   ********************************************************************
\documentclass[11pt]{article}

\usepackage[T1]{fontenc}
\usepackage{mathptmx}
\topmargin 0.0in
\setlength{\textwidth} {420pt}
\setlength{\textheight} {620pt} 
\setlength{\oddsidemargin} {20pt}
\setlength{\marginparwidth} {72in}

%   ********************************************************************
%   * Many of the commands below were simply copied over from an older *
%   * version of the proposal template; you can just leave them as     *
%   * they are (or you can delve into the TeX/LaTeX documentation      *
%   * and figure out what they do). Otherwise, jump ahead to the next  *
%   * block of comments, where you will enter title, abstract, etc.    *
%   ********************************************************************

\usepackage{fancyhdr} 
\usepackage{url}
\usepackage{graphicx}

% set it so that subsubsections have numbers and they
% are displayed in the TOC (maybe hard to read, might want to disable)

\setcounter{secnumdepth}{3}
\setcounter{tocdepth}{3}

% define widow protection

\def\widow#1{\vskip #1\vbadness10000\penalty-200\vskip-#1}

\clubpenalty=10000  % Don't allow orphans
\widowpenalty=10000 % Don't allow widows

% this should give me the ability to use some math symbols that 
% were available by default in standard latex (i.e. \Box)

\usepackage{latexsym}
\usepackage{color}
% define a little section heading that doesn't go with any number

\def\littlesection#1{
\widow{2cm}
\vskip 0.5cm
\noindent{\bf #1}
\vskip 0.0001cm 
}

\pagestyle{fancyplain}

\newcommand{\tstamp}{\today}   
\renewcommand{\sectionmark}[1]{\markright{#1}}
\lhead[\Section \thesection]            {\fancyplain{}{\rightmark}}
\chead[\fancyplain{}{}]                 {\fancyplain{}{}}
\rhead[\fancyplain{}{\rightmark}]       {\fancyplain{}{\thepage}}
\cfoot[\fancyplain{\thepage}{}]         {\fancyplain{\thepage}{}}

\newlength{\myVSpace}% the height of the box
\setlength{\myVSpace}{1ex}% the default, 
\newcommand\xstrut{\raisebox{-.5\myVSpace}% symmetric behaviour, 
  {\rule{0pt}{\myVSpace}}%
}

% leave things with no spacing extra spacing in the final version of the paper
\renewcommand{\baselinestretch}{1.0}    % must go before the begin of doc

% suppress the use of indentation for a paragraph

\setlength{\parindent}{0.0in}
\setlength{\parskip}{0.1in}





\begin{document}


% handle widows appropriately
\def\widow#1{\vskip #1\vbadness10000\penalty-200\vskip-#1}

% build the title section

\makeatletter

\def\maketitle{%
  %\null
  \thispagestyle{empty}%
  %\vfill
  \begin{center}%\leavevmode
    %\normalfont
    {\Huge \@title\par}%
    %\hrulefill\par
    {\normalsize \@author\par}%
    \vskip .4in
%    {\Large \@date\par}%
  \end{center}%
  %\vfill
  %\null
  %\cleardoublepage

  }

\makeatother

%   ********************************************************************
%   * Here is the first place where you need to begin customizing:     *
%   * Enter you name, the title of your proposal, etc., in the places  *
%   * indicated by the comment "% CHANGE!".                            *
%   ********************************************************************

\vspace*{-1.1in}
\title{My Report}% code}  % CHANGE!

% build the author section
\author{
        Oliver Bonham-Carter\\  % CHANGE!
        Department of Computer Science\\
        Allegheny College \\
        {\tt youremail@allegheny.edu}  \\  % CHANGE!
        \url{http://www.cs.allegheny.edu/~yourwebsite/} \\  % CHANGE OR DELETE!
        \vspace*{.1in} \today \\ \vspace*{.1in}
}

\maketitle       % use the default title stuff





%%%%%%%%%%%%%%%%%%%%%%%%%%%
% import each section into your doc!
% import each section into your doc!
% import each section into your doc!
% import each section into your doc!
% import each section into your doc!
% import each section into your doc!
% import each section into your doc!
% import each section into your doc!
% import each section into your doc!
% import each section into your doc!
% see code below !
%%%%%%%%%%%%%%%%%%%%%%%%%%%
%\thispagestyle{empty}


% Thesis Abstract -----------------------------------------------------

\begin{abstract}        %this creates the heading for the abstract page
%\begin{center}
%{\large{\textbf{Add bold wording here... }}}\\
%\vspace{0.15in}
%\end{center}



%Abstract has no more than about 295 words

Abstract details. Abstract details. Abstract details. Abstract details. Abstract details. Abstract details. Abstract details. Abstract details. Abstract details. Abstract details. Abstract details. 

\end{abstract}

 % abstract

\section{Introduction}
%\begin{center}
%{\large{\textbf{Add bold wording here... }}}\\
%\vspace{0.15in}
%\end{center}



%Abstract has no more than about 295 words
\color{red}
Introduction details. MY FACT AND REFERENCE IS HERE \cite{yu2018qphos}! 

Introduction details. 
My other reference is here \cite{atkins2018computer}. Introduction details. Introduction details. Introduction details. Introduction details. Introduction details. ANOTHER REFERENCE \cite{conrad-gecco-selection-study}. Introduction details. Introduction details. Introduction details. Introduction details. Introduction details. 

\color{black}
Introduction details. MY FACT AND REFERENCE IS HERE \cite{yu2018qphos}! 

\color{green}
Introduction details. 
My other reference is here \cite{atkins2018computer}. Introduction details. Introduction details. Introduction details. Introduction details. Introduction details. ANOTHER REFERENCE \cite{conrad-gecco-selection-study}. Introduction details. Introduction details. Introduction details. Introduction details. Introduction details. 
\color{black}

Introduction details. MY FACT AND REFERENCE IS HERE \cite{yu2018qphos}! 

Introduction details. 
My other reference is here \cite{atkins2018computer}. Introduction details. Introduction details. Introduction details. Introduction details. Introduction details. ANOTHER REFERENCE \cite{conrad-gecco-selection-study}. Introduction details. Introduction details. Introduction details. Introduction details. Introduction details. 


Introduction details. MY FACT AND REFERENCE IS HERE \cite{yu2018qphos}! 

Introduction details. 
My other reference is here \cite{atkins2018computer}. Introduction details. Introduction details. Introduction details. Introduction details. Introduction details. ANOTHER REFERENCE \cite{conrad-gecco-selection-study}. Introduction details. Introduction details. Introduction details. Introduction details. Introduction details. 


 % introduction section

\section{Related Works}
%\begin{center}
%{\large{\textbf{Add bold wording here... }}}\\
%\vspace{0.15in}
%\end{center}

This is my Related Works section. This is my Related Works section. This is my Related Works section. This is my Related Works section. This is my Related Works section. This is my Related Works section. This is my Related Works section. This is my Related Works section. This is my Related Works section. This is my Related Works section. This is my Related Works section. This is my Related Works section. This is my Related Works section. This is my Related Works section. This is my Related Works section. This is my Related Works section. This is my Related Works section. This is my Related Works section. This is my Related Works section. This is my Related Works section. This is my Related Works section. This is my Related Works section. This is my Related Works section. This is my Related Works section. This is my Related Works section. This is my Related Works section. This is my Related Works section. This is my Related Works section. This is my Related Works section. This is my Related Works section. This is my Related Works section. This is my Related Works section. 


 % related works 


% place other sections here...

% note: Be sure that your completed document resembles (exactly) the format used by Allegheny College's senior thesis documents



\bibliographystyle{plain}
\bibliography{bibliography}

\end{document}

